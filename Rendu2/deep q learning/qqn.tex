% Packages
\documentclass[11pt,a4paper]{article}
\usepackage[utf8]{inputenc}
\usepackage[T1]{fontenc}
\usepackage{hyperref}
\usepackage{lmodern}
\usepackage[english]{babel}
\usepackage{appendix}
\usepackage{enumitem}

%Fusionneer des pages
\usepackage{pgfpages}
%\pgfpagesuselayout{4 on 1}[a4paper,border shrink=5mm]

% Taille marges
\usepackage[left=2cm,right=2cm,top=2cm,bottom=2cm]{geometry}
% Titles size
\usepackage[small]{titlesec}

% math
\usepackage{amsfonts}
\usepackage{amsmath}
\usepackage{amssymb}
\usepackage{mathabx}
\usepackage{stmaryrd}

\usepackage[most]{tcolorbox}
\newtcolorbox[auto counter]{definition}[1]{colframe=red!75!black, coltitle=white, enhanced, frame empty, colback=white,fonttitle=\bfseries , title=Def \thetcbcounter$\;$: #1, borderline west={2pt}{0pt}{red!85!black},
attach boxed title to top left={xshift=-5mm}, boxed title style={colback=red!75!black}}

\newtcolorbox[auto counter]{prop}{colframe=black!80!white, coltitle=black, enhanced, frame empty, colback=white,fonttitle=\bfseries , title=\underline{Property \thetcbcounter$\;$:}, borderline west={2pt}{0pt}{black},
attach boxed title to top left={xshift=-4mm}, boxed title style={frame empty,colback=white}}

\newtcolorbox[auto counter]{thm}[1]{colframe=blue!70!black,colback=white,fonttitle=\bfseries , title=Theorem \thetcbcounter$\;$: #1}

\newtcolorbox[auto counter]{exercice}{colframe=white,colback=white,fonttitle=\bfseries , title=Exercice \thetcbcounter$\;$:}

\newtcolorbox{preuve}{boxrule=0pt, enhanced, colback=white, colframe=white, coltitle=black, fonttitle=\bfseries , title=\underline{Proof $\;$:},
top=0mm, frame empty, borderline west={1pt}{0pt}{black}, sharp corners,
after upper={\par\hfill\textit{$\blacksquare $}}}

\newtcolorbox{mybox}{colframe=white!75!black,colback=white!95!black,fonttitle=\bfseries}

% pseudo code
\usepackage[ruled,lined,noend]{algorithm2e}
\usepackage{babel}

% insertion image
\usepackage{graphicx}
\graphicspath{ {./images/} }

% derivation tree
\usepackage{ebproof}

% automate
\usepackage{caption}
\usepackage{tikz}
\usetikzlibrary{automata, positioning, arrows, decorations.pathreplacing, decorations.markings, positioning, shapes, quotes}


\newcounter{fig}
\newcommand{\fig}[3]{
	\begin{center}
	\begin{figure}[ht]
		\refstepcounter{fig}
		\centering
		\begin{tikzpicture}[scale=#3]
		#1
		\end{tikzpicture}
		\caption{\underline{#2}}
	\end{figure}
	\end{center}
}

\newcommand{\tab}{\phantom{xxx}}

\newcommand{\ignore}[1]{}

\newcommand{\uao}[3]{\underset{#1}{\overset{#2}{#3}}}

\renewcommand{\lim}[2]{\underset{#1 \rightarrow #2}{lim}}

\newcommand{\mlist}[1]{\begin{itemize}[noitemsep,topsep=0pt]#1\end{itemize}}



\title{\vspace{-1.0cm}Performance Evaluation project:\\\underline{Optimizing cars' trajectory with AI}}
\date{}
\author{\vspace{-1cm}Ottavy Macéo, Longatte Mathieu, Louison Mocq}

\begin{document}


    Deep Q-Learning is a reinforcement learning algorithm that combines Q-Learning with Deep Learning to solve complex decision-making problems. 
    It allows an agent to learn how to act optimally in environments with large state spaces by approximating a function, known as the \textit{Q-function}, which evaluates the quality of an action taken in a given state.
    
    \subsubsection*{Q-function}
    The Q-function, $Q(s, a)$, represents the expected cumulative reward an agent will receive after taking action $a$ in state $s$, and then following the optimal policy. The cumulative reward is computed as:
    \[
    Q(s, a) = r + \gamma \max_{a'} Q(s', a'),
    \]
    Where:
    \begin{itemize}
        \item $r$ is the immediate reward received after taking action $a$ in state $s$.
        \item $s'$ is the next state reached.
        \item $a'$ is the next action.
        \item $\gamma \in [0, 1]$ is the discount factor, which balances immediate and future rewards.
    \end{itemize}
    
    % \subsection*{Deep Q-Learning}
    % In DQL, a deep neural network, called the \textit{Q-network}, is used to approximate the Q-function. This network takes the state $s$ as input and outputs estimated Q-values for all possible actions. By doing so, DQL can handle environments with high-dimensional or continuous state spaces.
    
    \subsubsection*{Key Techniques}
    \begin{itemize}
        \item \textbf{Replay Buffer}: A memory that stores past experiences $(s, a, r, s')$. Randomly sampling experiences from the buffer during training reduces correlations between consecutive samples, improving learning stability.
        \item \textbf{Exploration-Exploitation Balance}: The agent uses an $\epsilon$-greedy policy to choose actions, where it explores randomly with probability $\epsilon$ and exploits the best-known action otherwise.
    \end{itemize}
    
    \subsubsection*{High-Level Workflow}
    \begin{enumerate}
        \item Observe the current state $s$.
        \item Choose an action $a$ using an $\epsilon$-greedy policy.
        \item Execute the action, observe the reward $r$ and next state $s'$.
        \item Store the experience $(s, a, r, s')$ in the replay buffer.
        \item Sample a mini-batch of experiences from the buffer to train the Q-network.
    \end{enumerate}
    

\end{document}